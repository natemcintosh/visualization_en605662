\documentclass{article}
\usepackage[utf8]{inputenc}
\usepackage{booktabs}
\usepackage[margin=1in]{geometry}
\usepackage{hyperref}
\hypersetup{
    colorlinks=true,
    linkcolor=blue,
    filecolor=magenta,
    urlcolor=blue,
}

\urlstyle{same}

\title{Project 4 - Interactive Visualization using D3}
\author{Nathan McIntosh}
\date{Due November 3, 2020}

\usepackage{natbib}
\usepackage{graphicx}

\begin{document}

\maketitle

\textbf{NOTE: This is a robust assignment. I strongly encourage you to begin work early. I will extend my office hours during this assignment and am also happy to take meetings by request. I highly encourage you to reach out to me for help with data set identification and dashboard concept as your submissions will be assessed against examples provided by each platform as well as public github repositories.}

\section{Purpose}
During the last 5 years, a number of software tools have been designed to help users explore complex data by creating visualizations and dashboards. Popular off-the-shelf tools include Tableau, Qlikview, Spotfire, Microsoft Power BI, MicroStrategy, Birst, and Logi among many others. Unfortunately, those tools only provide a small set of built-in visualizations that the user must use to visualize the data.  Often users are interested in using advanced visualization techniques that are not available within those off-the-shelf applications.

Recently, a number of libraries have been released to help organizations develop new visualizations and illustration tools.  Some of the popular libraries include D3, Processing, R, and ProtoVIS.   As data scientists, it is important for us to have a basic understanding of those libraries and their capabilities.  The purpose of this assignment is to get familiar with open source libraries by developing three sample visualizations using D3.js, NVD3.js, Plot,ly, or R Shiny.

\section{Task}
\begin{enumerate}
    \item Find 3 datasets that you would like to analyze.  The data should have over 100 rows and one of them should have more than 5 variables. \textit{Datasets in D3, NVD3, Plot.ly or R Shiny, or any other visualization platforms cannot be used in the project.}
    \item Develop three different visualizations in D3, NVD3, Plot.ly, or R Shiny to illustrate the datasets that you selected.

    \begin{itemize}
        \item Students should develop their own visualizations or select samples from the corresponding libraries

        \begin{itemize}
            \item D3 Gallery (https://github.com/d3/d3/wiki/Gallery)
            \item NVD3 Gallery (http://nvd3.org/examples/)
            \item R Shiny Gallery (https://shiny.rstudio.com/gallery/)
            \item Plot.ly Gallery (https://plot.ly/python/)
        \end{itemize}

        \item Students must change the data source to use the dataset they selected as part of step 1.

        \item Students should make at least one significant change to each visualization / source code they selected. Changes can be (but not limited to) changing colors, adding tooltips, adding interaction, incorporating D3 within a Bootstrap framework, adding a markdown with analysis, etc… If students are modifying existing D3, NVD3, Plot.ly, or R Shiny examples, students must describe in the document the changes that they did to each of the original programs.

        \item Each of the visualizations must have some sort of user interaction to enable data exploration (e.g. filters, sorting, selection, tooltip, etc...).

    \end{itemize}

    \item File structure:  students should structure their code the following way.
    \begin{itemize}
        \item your\_lastname\_project04/
        \begin{itemize}
            \item your\_lastname\_project04.docx or your\_lastname\_project04.pdf
            \item JS/
            \item Sample01/
            \begin{itemize}
                \item Index1.html
                \item Data1.csv
                \item Screenshot\_sample01.jpg
            \end{itemize}


            \item Sample02/
            \begin{itemize}
                \item Index2.html
                \item Data2.csv
                \item Screenshot\_sample02.jpg
            \end{itemize}

            \item Sample03/
            \begin{itemize}
                \item Index3.html
                \item Data3.csv
                \item Screenshot\_sample02.jpg
            \end{itemize}

        \end{itemize}
    \end{itemize}
    or something like

    \begin{itemize}
        \item your\_lastname\_project04/
        \begin{itemize}
            \item your\_lastname\_project04.docx or your\_lastname\_project04.pdf
            \item R/
            \item Sample01/
            \begin{itemize}
                \item Index1.html
                \item Sample01.R
                \item Data1.csv
                \item Screenshot\_sample01.jpg
            \end{itemize}


            \item Sample02/
            \begin{itemize}
                \item Index2.html
                \item Sample02.R
                \item Data2.csv
                \item Screenshot\_sample02.jpg
            \end{itemize}

            \item Sample03/
            \begin{itemize}
                \item Index3.html
                \item Sample03.R
                \item Data3.csv
                \item Screenshot\_sample02.jpg
            \end{itemize}

        \end{itemize}
    \end{itemize}
    If using JavaScript, students must include a directory with the JavaScript files.


\end{enumerate}

\section{Useful Links}
\begin{itemize}
    \item https://d3js.org/
    \item https://github.com/d3/d3/wiki/Tutorials
    \item http://alignedleft.com/tutorials/d3/
    \item https://shiny.rstudio.com
    \item http://nvd3.org
    \item https://plot.ly/python/
    \item https://jupyter-dashboards-layout.readthedocs.io/en/latest/using.html
\end{itemize}

\section{What to submit}
\begin{itemize}
    \item A .zip file with the file structure shown above
    \item Paper describing your project, the datasets that were chosen, the thee visualizations that were developed (including screenshot), and explanation about what was updated from any sample code that was used. 
    \item Submit document through Blackboard.  Please use the following file format: your\_lastname\_project04.zip

\end{itemize}

\end{document}
