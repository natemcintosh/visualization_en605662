\documentclass{article}
\usepackage[utf8]{inputenc}
\usepackage{booktabs}
\usepackage[margin=1in]{geometry}
\PassOptionsToPackage{hyphens}{url}
\usepackage{hyperref}
\hypersetup{
    colorlinks=true,
    linkcolor=blue,
    filecolor=magenta,      
    urlcolor=blue,
}

\urlstyle{same}

\title{Final Project - Proposal and Project Plan}
\author{Nathan McIntosh}
\date{Due October 26, 2020}


\begin{document}

\maketitle

\section{A description of your topic}
Something I've increasingly come to appreciate is the \href{https://brew.sh/}{Homebrew} package manager for Mac OS and Linux. Often I find myself wanting to test out a new software tool, or make sure that a setup ``just works''. It allows for easy uninstalling of packages no longer needed, and easy upgrading and downgrading of versions. As I watch it install, update, remove, etc.. packages, I'm always interested to see what dependencies are required, upgraded, downgraded. With the command \texttt{brew info <package name>} one can see statistics about the package. On my mac, \href{https://dotnet.microsoft.com/}{Microsoft's .NET} platform has the dependencies \href{https://cmake.org/}{cmake}, \href{https://www.freedesktop.org/wiki/Software/pkg-config/}{pkg-config}, \href{https://curl.haxx.se/}{curl}, \href{http://site.icu-project.org/}{icu4c}, and \href{https://www.openssl.org/}{openssl}. At the time of writing, it has the following statistics:
\begin{itemize}
    \item install: 622 (30 days), 676 (90 days), 678 (365 days)
    \item install-on-request: 414 (30 days), 468 (90 days), 470 (365 days)
    \item build-error: 0 (30 days)
\end{itemize}
From this we can clearly see that either .NET is not yet used on a large scale on Mac OS, or Homebrew is not the primary means by which it is installed. What could we learn if we looked at this information for all the most popular packages Homebrew serves on both Mac and Linux? What I hope to accomplish in my final project is an overview of the state of Homebrew dependencies on Mac and Linux. Which are asked for most often? Which are downloaded as dependencies most often? How are popular packages interconnected? Which versions of Mac OS are being used the most?

\section{A detailed plan of how the project will get done}
I plan to use Homebrew's excellent \href{https://formulae.brew.sh/analytics/}{analytics} APIs to get the relevant information. This includes
\begin{itemize}
    \item Mac OS formula install events 
    \item Mac OS formula install on request events
    \item Mac OS formula build error events
    \item Mac OS cask install events
    \item Mac OS versions for events
    \item Linux formula install events
    \item Linux formula install on request events
    \item Linux formula build error events
\end{itemize}
Furthermore, there is an API to get all information for each formula (also split into Mac and Linux), which gives the statistics again, but also information about dependencies, conflicts, etc. Documentation for all of these APIs is available \href{https://formulae.brew.sh/docs/api/}{here}.

To get the data in a format easy for plotting, I'll write Python functions that consume the JSON produced by the APIs and produce pandas dataframes. I will then write a primary plotting function which will use the Plotly plotting library to produce an HTML file of the visualization(s).

\section{A detailed time line}
Breaking up the remaining time by module:
\begin{itemize}
    \item Module 9: Write Python functions to successfully pull down JSON data about install and error events, and transform into pandas dataframes. 
    \item Module 10: Revise project proposal. Write Python functions to get individual package JSON info and transform into dataframes.
    \item Module 11: Write literature survey. Write Python functions to combine all collected data into smallest possible number of dataframes. 
    \item Module 12: Sketch examples of possible visualizations with tools like PowerPoint. Begin writing plotting code. 
    \item Module 13: Finish writing plotting code. Put together documentation about how to run the code, and sample data that can be used to test the system. Put together outline of the final paper, with a few very rough sentences or figures that get the idea across for each section.
    \item Module 14: Revise final paper. Submit paper and project. 
\end{itemize}

\section{Your initial ideas for which course concepts are important to your project}
\begin{itemize}
    \item Network and graph visualization will be important to see how interconnected packages are within the Mac OS and Linux ecosystems. 
    \item Human visual perception will be important to understand to best communicate any differences between Linux and Mac.
    \item I think it'll be easy to make visualizations for this topic that human visual perception struggles to intuitively grasp
    \item An important part of this project will be the ability to interact with the data; look at specific packages, highlight linked packages, etc.
\end{itemize}

\section{A description of how your project differs from existing implementations, if any exist, of these data}

I have so far been unable to find any existing visualizations of this set of data. Some visualizations do exist for dependencies of various pieces of software. Here are links to a few:

\begin{itemize}
    \item \url{https://linkurio.us/blog/manage-software-dependencies-with-graph-visualization/}
    \item \url{https://www.netlify.com/blog/2018/08/23/how-to-easily-visualize-a-projects-dependency-graph-with-dependency-cruiser/}
    \item \url{https://www.ndepend.com/}
\end{itemize}

As far as I can tell, most of the existing implementations focus on the dependencies of individual packages/tools. While my project will look at dependencies of individual packages, it will also examine at the ecosystem as a whole: package popularity,  reliability of packages, and the effect of packages that do not build properly.

\section{A description of how your project enables better understanding of the data}
My project will give a complete overview of the most popular packages in the Mac OS and Linux ecosystems, as far as Homebrew is used to install software and manage dependencies. It will allow users to answer questions like ``what happens if this packages suddenly stops working for everyone?'', ``which packages have seen the most growth in the last year?'', or ``which packages are most popular and are staying most popular?''.


\end{document}
